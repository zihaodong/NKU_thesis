% -*- coding: utf-8 -*-
%%
%%
%%
%%
%%
%%
%%  本模板使用以下方式编译:
%%
%%
%%     1. XeLaTeX [推荐]\emph{•}
%%
%%  注意:
%%    1. 在改变编译方式前应先删除 *.toc 和 *.aux 文件,
%%       因为不同编译方式产生的辅助文件格式可能并不相同。
%%
%%
\documentclass[12pt,openany]{book}%openany
\usepackage{ccmap}
\usepackage{algorithm}
\usepackage{algorithmic}
\usepackage{setspace}
\usepackage{multirow}
\usepackage{amsmath}
%\allowdisplaybreaks[4]
\usepackage{ifxetex}
\ifxetex
  \usepackage[bookmarksnumbered]{hyperref}
\else
  \usepackage[unicode,bookmarksnumbered]{hyperref}
\fi

%%%新增 GBT 2015
%\usepackage[comma,square,super]{natbib}
\usepackage{natbib}
\usepackage{gbt7714}
%\bibliographystyle{gbt7714}
\usepackage[emptydoublepage]{NKThesis}   % 中文
%\usepackage[emptydoublepage,English]{NKThesis} % 英文
%\makeatletter
%\let\c@lofdepth\relax
%\let\c@lotdepth\relax
%\makeatother
%\usepackage{graphicx}
%\usepackage{subcaption}

\floatname{algorithm}{算法}
\renewcommand{\algorithmicrequire}{\textbf{输入:}}
\renewcommand{\algorithmicensure}{\textbf{输出:}}

\graphicspath{{figure/}}
% 根据需要选择 biblatex 宏包选项.
%\usepackage[backend = biber, defernumbers = true,  sorting=none,  style = nkthesis]{biblatex}
\hypersetup{colorlinks=true,
            pdfborder=0 0 1,
            citecolor=black,
            linkcolor=black}
%\usepackage{tikz}

%\bibliographystyle{gbt-7714-2015-author-year}
%\setcitestyle{numbers,square,comma,sort&compress}

\newcommand{\scite}[1]{\textsuperscript{\cite{#1}}}
%重定义\scite使得参考文献应用成为上标

%\newcommand{\scite}[1]{{\setcitestyle{square,super}\cite{#1}}}


%指定参考文献文件的名称,全名.bib
%\addbibresource{nkthesis.bib}
%新建条目分类(category)用于区分被引用和未引用的文献条目
%\DeclareBibliographyCategory{cited}
%每执行一次除\nocite之外的\cite类命令,将被被引用的文献加到‘cited’分类中
%\AtEveryCitekey{\addtocategory{cited}{\thefield{entrykey}}}

\includeonly{
abstract,
manual,
%tikz,
%references,
acknowledgements,
% appendices,
resume
}
\usepackage{ntheorem}
\theoremheaderfont{\jiacu\songti}
\theorembodyfont{\songti}
\newtheorem{Theorem}{\hskip 2em  定理}[chapter]
\newtheorem{Lemma}[Theorem]{\hskip 2em 引理}
\newtheorem{Corollary}[Theorem]{\hskip 2em 推论}
\newtheorem{Proposition}[Theorem]{\hskip 2em 命题}
\newtheorem{Definition}[Theorem]{\hskip 2em 定义}
\newtheorem{Remark}[Theorem]{\hskip 2em 注}
\newtheorem{Example}[Theorem]{\hskip 2em 例}

\theoremstyle{nonumberplain}
\newtheorem{Proof}[Theorem]{\hskip 2em 证明:}

\begin{document}

%  设置基本信息
%  注意:  逗号`,'是项目分隔符. 如果某一项的值出现逗号, 应放在花括号内, 如 {,}
%
\NKTsetup{%
  论文题目(中文) = 南开大学研究生学位论文模板,
  副标题         = thesis template,
  论文题目(英文) = ,
  论文作者       = ,
  学号           = ,
  指导教师       = ,
  申请学位       = ,
  培养单位       = 计算机学院,
  学科专业       = 计算机科学与技术,
  研究方向       = ,
  研究方向       = ,
  答辩委员会主席 = ,
  评阅人1        = 匿名评审, %可以写三位评阅人姓名
  评阅人2        = ,  % 其他的评阅人姓名
  中图分类号     = ,
  中图分类号     = ,
  UDC            = ,
  学校代码       = 10055,
  密级           = 公开,
                   % 公开 | 限制 | 秘密 | 机密, 若为公开, 不填以下三项
  保密期限       = ,
  审批表编号     = ,
  批准日期       = ,
  论文完成时间   = 二〇二〇年五月,
  答辩日期       = ,
  论文类别       = 博士,
                   % 博士 | 学历硕士 | 硕士专业学位 | 高校教师 | 同等学力硕士
  院/系/所       = ,
  专业           = ,
  联系电话       = ,
  Email          = ,
  通讯地址(邮编) = ,
  备注           = }


% -*- coding: utf-8 -*-


\begin{zhaiyao}
%这里输入中文摘要。



%\newpage
%中文摘要ABF
\end{zhaiyao}

\begin{guanjianci}
%这里输入中文关键词。



\end{guanjianci}


\begin{abstract}
%这里输入英文摘要。




\end{abstract}


\begin{keywords}
%这里输入英文关键词。


\end{keywords}

\tableofcontents
\include{manual}
%\include{tikz}
%\include{references}
% -*- coding: utf-8 -*-

%\makeschapterhead{致谢}
\chapter*{致谢}




%\include{appendices}
% -*- coding: utf-8 -*-


\chapter*{个人简历 $\quad$ 在学期间发表的学术论文与研究成果}
%\chapter*{在学期间发表的学术论文与研究成果}
\section*{基本情况}




\section*{教育经历}



\section*{XX期间获奖情况}
\begin{enumerate}
\renewcommand{\labelenumi}{[\theenumi]}
\item 
\item 
\item 
\item 
\end{enumerate}

\section*{XX期间发表论文}
\begin{enumerate}
\renewcommand{\labelenumi}{[\theenumi]}
%\item Bell F K. A note on the irregularity of graphs[J]. Linear algebra and its applications, 1992, 161: 45-54.
%\item Bell F K. A note on the irregularity of graphs[J]. Linear algebra and its applications, 1992, 161: 45-54.
%\underline
%\item Neurocomputing (SCI检索,影响因子:4.072,SCI 2区)
%\item IEEE Access (SCI检索,影响因子:4.098,SCI 2区)
%\item 计算机辅助设计与图形学学报(EI检索,CCF A类期刊)
%\item International Conference on Neural Information Processing 2018 (ICONIP 2018). (EI检索,CCF C类会议)
%\item IEEE Annual Computer Software and Applications Conference 2019 (COMPSAC 2019). (EI检索,CCF C类会议)
%\item International Conference on Artificial Neural Networks 2019 (ICANN 2019). (EI检索,CCF C类会议)
%\item International Conference on Tools with Artificial Intelligence 2019 (ICTAI 2019). (EI检索,CCF C类会议)
\item 
\item
\end{enumerate}

\section*{XX期间参加的研究项目}
\begin{enumerate}
\renewcommand{\labelenumi}{[\theenumi]}
\item 
\item 
\item 
\end{enumerate}


\end{document}
